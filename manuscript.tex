% Options for packages loaded elsewhere
\PassOptionsToPackage{unicode}{hyperref}
\PassOptionsToPackage{hyphens}{url}
%
\documentclass[
]{article}
\usepackage{amsmath,amssymb}
\usepackage{iftex}
\ifPDFTeX
  \usepackage[T1]{fontenc}
  \usepackage[utf8]{inputenc}
  \usepackage{textcomp} % provide euro and other symbols
\else % if luatex or xetex
  \usepackage{unicode-math} % this also loads fontspec
  \defaultfontfeatures{Scale=MatchLowercase}
  \defaultfontfeatures[\rmfamily]{Ligatures=TeX,Scale=1}
\fi
\usepackage{lmodern}
\ifPDFTeX\else
  % xetex/luatex font selection
\fi
% Use upquote if available, for straight quotes in verbatim environments
\IfFileExists{upquote.sty}{\usepackage{upquote}}{}
\IfFileExists{microtype.sty}{% use microtype if available
  \usepackage[]{microtype}
  \UseMicrotypeSet[protrusion]{basicmath} % disable protrusion for tt fonts
}{}
\makeatletter
\@ifundefined{KOMAClassName}{% if non-KOMA class
  \IfFileExists{parskip.sty}{%
    \usepackage{parskip}
  }{% else
    \setlength{\parindent}{0pt}
    \setlength{\parskip}{6pt plus 2pt minus 1pt}}
}{% if KOMA class
  \KOMAoptions{parskip=half}}
\makeatother
\usepackage{xcolor}
\usepackage[margin=1in]{geometry}
\usepackage{longtable,booktabs,array}
\usepackage{calc} % for calculating minipage widths
% Correct order of tables after \paragraph or \subparagraph
\usepackage{etoolbox}
\makeatletter
\patchcmd\longtable{\par}{\if@noskipsec\mbox{}\fi\par}{}{}
\makeatother
% Allow footnotes in longtable head/foot
\IfFileExists{footnotehyper.sty}{\usepackage{footnotehyper}}{\usepackage{footnote}}
\makesavenoteenv{longtable}
\usepackage{graphicx}
\makeatletter
\def\maxwidth{\ifdim\Gin@nat@width>\linewidth\linewidth\else\Gin@nat@width\fi}
\def\maxheight{\ifdim\Gin@nat@height>\textheight\textheight\else\Gin@nat@height\fi}
\makeatother
% Scale images if necessary, so that they will not overflow the page
% margins by default, and it is still possible to overwrite the defaults
% using explicit options in \includegraphics[width, height, ...]{}
\setkeys{Gin}{width=\maxwidth,height=\maxheight,keepaspectratio}
% Set default figure placement to htbp
\makeatletter
\def\fps@figure{htbp}
\makeatother
\usepackage{soul}
\setlength{\emergencystretch}{3em} % prevent overfull lines
\providecommand{\tightlist}{%
  \setlength{\itemsep}{0pt}\setlength{\parskip}{0pt}}
\setcounter{secnumdepth}{-\maxdimen} % remove section numbering
\newlength{\cslhangindent}
\setlength{\cslhangindent}{1.5em}
\newlength{\csllabelwidth}
\setlength{\csllabelwidth}{3em}
\newlength{\cslentryspacingunit} % times entry-spacing
\setlength{\cslentryspacingunit}{\parskip}
\newenvironment{CSLReferences}[2] % #1 hanging-ident, #2 entry spacing
 {% don't indent paragraphs
  \setlength{\parindent}{0pt}
  % turn on hanging indent if param 1 is 1
  \ifodd #1
  \let\oldpar\par
  \def\par{\hangindent=\cslhangindent\oldpar}
  \fi
  % set entry spacing
  \setlength{\parskip}{#2\cslentryspacingunit}
 }%
 {}
\usepackage{calc}
\newcommand{\CSLBlock}[1]{#1\hfill\break}
\newcommand{\CSLLeftMargin}[1]{\parbox[t]{\csllabelwidth}{#1}}
\newcommand{\CSLRightInline}[1]{\parbox[t]{\linewidth - \csllabelwidth}{#1}\break}
\newcommand{\CSLIndent}[1]{\hspace{\cslhangindent}#1}
\setlength{\parindent}{4em}
\setlength{\parskip}{0em}
\ifLuaTeX
  \usepackage{selnolig}  % disable illegal ligatures
\fi
\IfFileExists{bookmark.sty}{\usepackage{bookmark}}{\usepackage{hyperref}}
\IfFileExists{xurl.sty}{\usepackage{xurl}}{} % add URL line breaks if available
\urlstyle{same}
\hypersetup{
  pdftitle={manuscript},
  hidelinks,
  pdfcreator={LaTeX via pandoc}}

\title{manuscript}
\author{}
\date{\vspace{-2.5em}}

\begin{document}
\maketitle

\hypertarget{manuscript-draft}{%
\section{Manuscript Draft}\label{manuscript-draft}}

\hypertarget{introduction}{%
\subsection{Introduction}\label{introduction}}

\hypertarget{soundscapes}{%
\subsubsection{Soundscapes}\label{soundscapes}}

Sound serves as the main sensory input for many species in the ocean.
Since sound travels over four times faster in water than in air and can
carry information over far distances (Urick, 1983), the use of sound has
evolved as a critical component of communication, navigation, and
foraging for many invertebrates, fishes, and cetaceans (Duarte et al.,
2021). Under the right conditions, sounds can travel across ocean basins
(Miksis-Olds, Martin, and Tyack, 2018).\\

The total collection of sounds in an environment is called a soundscape,
and these sources of sound are grouped into three broad categories:
biophonies, sound produced by animals like cetaceans, fishes, and
invertebrates; geophonies, natural sounds like rain, wind, and
earthquakes; and anthropophonies, human-made sounds from shipping
vessels, sonar, and oceanographic survey activities (Duarte et al.,
2021; Miksis-Olds et al., 2018). Every region of the ocean experiences
high spatio-temporal variation in sound levels and sources based on
traits such as properties of water, organism diversity, human
activities, and bathymetry (André et al., 2011). Because of their unique
sound signatures, ocean soundscapes can be characterized, measured, and
monitored as indicators of ecosystem health or species diversity (Haver
et al., 2019; Pijanowski, Farina, Gage, Dumyahn, and Krause, 2011;
Radford, Stanley, Tindle, Montgomery, and Jeffs, 2010; Weiss et al.,
2021).\\

When noise levels reach higher than normal levels, organisms can be
negatively impacted. Invertebrates and fishes may experience stunted
development, elevated stress responses, and physical injuries (L.
Weilgart, 2018), while cetaceans' behavior, calling patterns, physiology
and stress levels, and ability to locate prey or conspecifics are
impacted (L. S. Weilgart, 2007). Sensitivities and responses to noise
can vary among species and even individuals (Kunc, McLaughlin, and
Schmidt, 2016). While terrestrial ecosystem-level effects have been
well-documented (Buxton et al., 2017; Francis, Kleist, Ortega, and Cruz,
2012; Shannon et al., 2016), effects within aquatic ecosystems require
further studies (Kunc et al., 2016).\\

Ocean noise pollution is among the anthropogenic changes that have
recently grown to concerning levels (Hildebrand, 2009; Tyack, 2008). The
global ambient noise in the ocean is estimated to have increased by 12
decibels in the last few decades and has a close correlation to global
economic trends (Frisk, 2012). Ships have been increasing in terms of
gross tonnage and fleet size (Hildebrand, 2009) and now account for over
80\% of the global trade, with an estimated 1.4\% expected annual
increase in shipping through 2027 (UNCTAD, 2022). Commercial shipping
now accounts for most of the low-frequency anthropogenic noise in the
ocean (Hildebrand, 2009).\\

These noise-generating activities are categorized as either incidental
or deliberate (Chahouri, Elouahmani, and Ouchene, 2022; Chou, Southall,
Robards, and Rosenbaum, 2021). Deliberate activities such as active
sonar pingers, resource extraction from the seabed, seismic surveys, and
echosounders intentionally produce sound to measure an area and can
utilize both high and low frequencies (Duarte et al., 2021). Incidental
activities like dredging, offshore development, and shipping produce
sound as a by-product (Hawkins and Popper, 2017; Southall et al., 2017).
Anthropogenic noise can further be characterized by the frequency and
duration of the sound. For example, shipping occurs in all oceans and
produces a continuous, low-frequency (\textless200 Hz) sound that
travels far distances (Hildebrand, 2009). Sonar for military
surveillance or research such as seafloor mapping produces mid-frequency
sounds, but attenuates quicker, therefore impacting a smaller acoustic
space (Hildebrand, 2009; Richardson, Jr, Malme, and Thomson, 2013).
Noise has been shown to propagate differently based on static features
like bathymetry and depth (Richardson et al., 2013; Vagle, Burnham,
O'Neill, and Yurk, 2021), as well as dynamic variables like temperature,
salinity, and pH from a changing climate (Affatati, Scaini, and Salon,
2022; Kunc et al., 2016).\\

As soniferous animals, cetaceans have varying levels of sensitivity and
vulnerability to certain types of noise. Generally, mysticetes are more
affected by low-frequency noise produced by large shipping vessels
(Southall et al., 2017). Odontocetes produce and detect mid- to
high-frequency sounds and are more likely to interfere with human
activities at these frequencies. These generalizations appear to have
exceptions, as odontocete species like beaked whales have shown
avoidance behaviors in response to broadband and low frequency vessel
noise (Aguilar Soto et al., 2006; Pirotta et al., 2012). It remains
uncertain how short-term effects of chronic or acute noise pollution for
cetaceans translates into long-term or population level changes. New,
Moretti, Hooker, Costa, and Simmons (2013) suggest enough exposure can
cause beaked whales to cease foraging and eventually lower reproduction
rates.\\

\hypertarget{beaked-whales-and-sperm-whales}{%
\subsubsection{Beaked Whales and Sperm
Whales}\label{beaked-whales-and-sperm-whales}}

Beaked whales (family \emph{Ziphiidae)} are a group of 24 species and
among the least understood cetaceans and deepest divers in the ocean.
They regularly dive for over an hour to depths greater than 800 meters,
sometimes exceeding three hours and 2000 meters, followed by only a few
minutes on the surface (Baird et al., 2006; Tyack, Aguilar Soto,
Johnson, Sturlese, and Madsen, 2006). Their preference for deep water,
coupled with a low-profile body makes them nearly invisible on the
surface, so very little is known about them with visual observations
(Awbery, 2022). Much of what is known comes from strandings and stomach
content analysis (West, Walker, Baird, Mead, and Collins, 2017). Even
when sighting events occur, they risk misidentification due to the
paucity of data of morphological variations in some species (Aguilar de
Soto et al., 2017). Tagging studies prove useful, but it is often
difficult to acquire a sizable dataset (Baird, 2008; Baird et al., 2006;
Tyack et al., 2006).

Beaked whales are known to be exceptionally sensitive to noise (Hooker
et al., 2019). There are several documented cases of naval mid-frequency
active sonar (MFAS) activities coinciding with mass stranding events by
beaked whales (Bernaldo de Quiros et al., 2019; DeRuiter et al., 2013;
Fernández et al., 2005; Simonis, Brownell, et al., 2020; Stanistreet et
al., 2022; L. S. Weilgart, 2007). The same physiological adaptations
that allow for extreme diving are observed to increase their
decompression risk when they alter their behavior in response to noise
(Fahlman, Tyack, Miller, and Kvadsheim, 2014; Kvadsheim et al., 2012).
Furthermore, these responses vary based on the individuals' age, sex,
past experience with noise, or current activity (L. S. Weilgart, 2007).

Despite their visually cryptic nature, their sound signals are very
well-described, down to the species level (Baumann-Pickering et al.,
2013; Baumann-Pickering et al., 2014). Beaked whales produce clicks,
pulses, and buzzes during foraging below 500 meters of depth (Tyack et
al., 2006). Increased acoustic surveys reported in the literature have
detected new vocalization patterns globally (Manzano-Roth et al., 2023),
with notable temporal patterns emerging (Yamada et al., 2019).

Sperm whales (\emph{Physeter macrocephalus}) also occupy deep ocean
habitats, but their dive depths and times are reduced in comparison.
Sperm whales have been known to dive regularly between 400-600 meters
and often over 1000 meters (Amano and Yoshioka, 2003; Awbery, 2022;
Watkins, Daher, Fristrup, Howald, and Di Sciara, 1993). They also appear
to have more acoustic tolerance than beaked whales (Madsen, Mohl,
Nielsen, and Wahlberg, 2002; Patrick J. O. Miller et al., 2022; P. J. O.
Miller et al., 2009; Winsor, Irvine, and Mate, 2017). While avoidance
responses in sperm whales have been recorded following MFAS events (Curé
et al., 2013; Sivle et al., 2012; Stanistreet et al., 2022) and seismic
survey pulses (Madsen et al., 2002) , and can range from minor (Isojunno
et al., 2022) to moderate (Isojunno et al., 2016) in severity, the
circumstances surrounding the noise exposure is not well understood.

Sperm whales also emit series of targeted clicks when socializing (Fais
et al., 2015; Marcoux, Whitehead, and Rendell, 2006) and foraging (Fais,
Johnson, Wilson, Aguilar Soto, and Madsen, 2016), but it's unlikely they
debilitate their prey with their vocalizations as previously
hypothesized (Norris and Harvey, 1972). More is known about sperm whales
than beaked whales as they spend more time on the surface, lending
themselves to better detection during visual surveys.

Several studies to date have documented specific effects of acoustic
interference from human activity in beaked and sperm whales. Intense
noise events can affect behavior through habitat displacement, time and
effort fleeing spent instead of foraging, and communication and
echolocation click disruption (Aguilar Soto et al., 2006; Cholewiak,
DeAngelis, Palka, Corkeron, and Van Parijs, 2017; DeRuiter et al., 2013;
Falcone et al., 2017; Joyce et al., 2020; Stanistreet et al., 2022);
cause physiological harm like hearing loss, tissue damage, and elevated
stress levels (Cox et al., 2006; Hooker et al., 2019) even up to 100 km
away (Falcone et al., 2017); and mask signals for communication (Erbe,
Williams, Sandilands, and Ashe, 2014), environmental cues (Duarte et
al., 2021), or mating (L. S. Weilgart, 2007).

Under the Marine Mammal Protection Act, these species are federally
protected, and several receive addition protection under the Endangered
Species Act due to a vulnerable or endangered conservation status
(Taylor et al., n.d.), but some remain listed as ``data deficient'' by
the International Union for the Conservation of Nature (IUCN) Red List
of Threatened Species (Pitman and Brownell, n.d., 2020). The latter
designation has been scrutinized for not eliciting the same conservation
urgency as species whose populations are well-documented (Parsons,
2016), therefore emphasizing the need for population and abundance
studies of these species.

\hypertarget{passive-acoustic-monitoring}{%
\subsubsection{Passive Acoustic
Monitoring}\label{passive-acoustic-monitoring}}

Passive acoustic monitoring (PAM) is a tool that allows researchers to
record and listen to the soundscape. It is also an ideal method for
assessing habitat use and improving species distribution models of
visually cryptic cetaceans like beaked whales and sperm whales that are
more likely to be heard than seen (Fleishman et al., 2023), live
relatively far offshore (Robbins, Bell, Potts, Babey, and Marley, 2022),
and have low-density populations (Arranz et al., 2023; Hildebrand et
al., 2015; Hooker et al., 2019).

Passive acoustic recorders come in two forms: fixed or mobile, each
better suited to a certain ecological application. A mobile (towed or
free drifting) buoy is either towed behind a vessel or moves
autonomously with the currents for days or weeks at a time. It captures
the dynamic nature of sound over a large spatial area, especially in
deep and less studied parts of the ocean, making it ideal for detecting
offshore or mobile species. Bottom-mounted recorders allow for greater
temporal coverage and are often used when studying specific regions or
known habitats of study species (Jarvis, DiMarzio, Watwood, Dolan, and
Morrissey, 2022; Marques et al., 2012; Nosal and Frazer, 2007).

The benefits of using PAM over visual surveys include use during night
hours or through inclement weather, lower cost to charter and little
experience needed to conduct a survey (Fleishman et al., 2023; Johnson,
Soto, and Madsen, 2009). The trade-offs include costly equipment or
deployment, high data/battery storage requirements, uncertain chance of
return, limited spatial coverage for fixed recorders, and limited
temporal coverage for mobile recorders. Even with experienced
acousticians, noisy detections can obfuscate number of individuals or
species detected and distance or direction from recorder for species
with large call ranges such as sperm whales (Barlow and Gisiner, 2005;
Mellinger, Stafford, Moore, Dziak, and Matsumoto, 2007).

\hypertarget{automatic-identification-system}{%
\subsubsection{Automatic Identification
System}\label{automatic-identification-system}}

Automatic Identification System (AIS) is a mandatory vessel tracking
system for vessels over 300 gross tons to aid in position communication
and safe navigation (Federal Register, 2003). While not originally
intended for use in research, it has become a valuable resource for
vessel traffic data, especially in studying effects of vessel strikes
(Greig, Hines, Cope, and Liu, 2020; Reimer, Gravel, Brown, and Taggart,
2016) and noise pollution (Aguilar Soto et al., 2006; Erbe et al., 2014)
on cetaceans. Because ships are abundant throughout the study area due
to major U.S. ports, AIS was included in this study to investigate
patterns of co-occurrence between vessels and whales and sound levels in
those moments. Previous studies have separated vessel and anthropogenic
noise from ambient soundscape based on known spectral signatures,
frequency bands, and percentiles (Haver et al., 2020; Weiss et al.,
2021). The inclusion of AIS to soundscape metrics in this study provides
fine-scale evaluation of sound and source.

\hypertarget{laws-and-regulations}{%
\subsubsection{\texorpdfstring{\textbf{Laws and
Regulations}}{Laws and Regulations}}\label{laws-and-regulations}}

While there is consensus that anthropogenic noise in the ocean is
increasing, policies that aim to mitigate chronic noise are generally
lacking. This is due to knowledge gaps regarding the long-term and
cumulative effects of noise on certain species (Markus and Sánchez,
2018). International and regional efforts are underway (see Chou et al.
(2021)) but are not fully complete. Within the European Union, the
Marine Strategy Framework Directive establishes a directive about
environmental regulation on marine noise, specifically requiring member
states to adopt strategies in compliance with threshold values for
impulsive and continuous noise (Borsani, J.F., Juretzek C., Klauson A.,
Leaper R., Le Courtois F., Liebschner A., Maglio A., Mueller A. , Norro
A., Novellino A., Outinen O., Popit A., Prospathopoulos A., Sigray P.,
Thomsen F., Tougaard J., Vukadin P., and Weilgart L., Borsani, J.F.,
Andersson M., André M., Azzellino A., Bou M., Castellote M., Ceyrac L.,
Dellong D., Folegot T., Hedgeland D., and Juretzek C., Klauson A.,
Leaper R., Le Courtois F., Liebschner A., Maglio A., Mueller A. , Norro
A., Novellino A., Outinen O., 2023; \emph{Directive 2008/56/EC of the
European Parliament and of the Council of 17 June 2008 establishing a
framework for community action in the field of marine environmental
policy (Marine Strategy Framework Directive) (Text with EEA relevance)},
2008). Some studies have independently tested these threshold values
(Garrett et al., 2016; Haver et al., 2021; Sebastianutto, Fortuna,
Mackelworth, Holcer, and Rako Gospić, 2015) or used them as guidelines
in their study (Mustonen et al., 2019). The National Oceanic and
Atmospheric Administration (NOAA) has developed an Ocean Noise Strategy
Roadmap, an agency-wise approach to mitigating both chronic and acute
effects of noise exposure to animals (NOAA, 2018). Within it, the
authors agree upon impairment thresholds for mid-frequency cetaceans
(NOAA, 2018). Because chronic effects are not fully understood, they
remain to be regulated. The International Maritime Organization (IMO)
has also been addressing noise pollution from commercial vessels through
quieter ship design (International Maritime Organization, 2014). Updates
to their 2014 guidelines are set to be finalized by January 2024.
Finally, Canada has launched a Quiet Vessel Initiative and is expected
to publish a Ocean Noise Strategy for underwater noise management in
2023 (Breeze et al., 2022).

\hypertarget{research-objective}{%
\subsubsection{Research Objective}\label{research-objective}}

With anthropogenic noise-generating activities on the rise, global
oceanic sound levels are expected to increase. Noise is a form of
habitat degradation, so there is a need for research on a large spatial
scale to understand how these highly mobile animals overlap in space and
time with human activities and how they're affected (Erbe et al., 2014;
Haver et al., 2019; Hooker et al., 2019). Acquiring information on these
federally protected and acoustically sensitive species is challenging,
but certain PAM techniques can be well-suited for this type of research.
Logistical difficulties associated with offshore research and
low-density populations of visually cryptic deep diving whales have
created a paucity of data in the California Current Ecosystem. This
research provides a step towards understanding soundscape variation in
the area and beaked and sperm whale distributions and their acoustic
environment.

The purpose of this study is to characterize the spatial and temporal
acoustic variability of beaked and sperm whale habitat in the California
Current Ecosystem using soundscape metrics, environmental variables, and
AIS. We determine significant predictors of whale presence for each
species with the use of Random Forest models and generalized additive
models and provide explicit sources for noise-generating human
activities through overlay of large vessel activity.

\hypertarget{methods}{%
\subsection{Methods}\label{methods}}

\hypertarget{study-area}{%
\subsubsection{\texorpdfstring{\textbf{Study
Area}}{Study Area}}\label{study-area}}

The California Current Ecosystem (CCE) is a highly productive and
complex ecosystem that contains one proposed and five current national
marine sanctuaries (Office of National Marine Sanctuaries, n.d.),
supports several wild fisheries (Field and Francis, 2006), and is home
to nineteen protected whale species (NOAA Fisheries, n.d.). The
1,141,800 km2 area extends along the US' Economic Exclusion Zone from
southern Canada to northern Mexico and is a characterized by deep
canyons and a continental shelf (Barlow et al., 2009; Barlow and Forney,
2007). Characteristic of the area are upwelling events that carry
nutrients to the surface and drive primary productivity, creating an
ideal habitat where whales' prey are found (Ryther, 1969). The year 2018
was characterized in the CCE with slightly warmer than average sea
surface temperatures as the recent El-Nino and marine heatwave events
were ending (Harvey et al., n.d.). Because this area has high economic
value and natural ecological activity, it's a well-managed and
well-researched area (Coleman, 2008; Oldach et al., 2022; Peña and
Bograd, 2007).\\
\strut \\
In 2021, California, Washington, and Oregon ranked 3rd, 5th, and 23rd in
most trafficked U.S. waterways, respectively (United States Army Corps
of Engineers {[}USACE{]}, 2018). Between vessel activity like fishing,
shipping and naval exercises in California alone, there is notable
overlap of anthropogenic presence with important habitat for protected
species. Furthermore, the future development of offshore wind turbines
off of Morro Bay and Humboldt, California are expected to be
noise-producing activities during construction and operation, increasing
sound levels in those regions (Cooperman et al., 2022; Madsen, Wahlberg,
Tougaard, Lucke, and Tyack, 2006).

\hypertarget{data-collection-and-processing}{%
\subsubsection{Data Collection and
Processing}\label{data-collection-and-processing}}

Acoustic detections were collected as part of the 2018 California
Current Ecosystem Survey (CCES) by the Southwest Fisheries Science
Center (SWFSC). Twenty-three free drifting buoys (numbered 1-23) were
deployed at varying times between July and November in the offshore
waters of the study area. These buoys contained two satellite
geo-locators above water to monitor GPS information and two hydrophones
at depth attached by a 150-meter line. They collected a total of 1910
hours of data. SWFSC detected and classified acoustic detections in
PAMGuard to species level as per methods mentioned in Simonis, Trickey,
et al. (2020).

Acoustic data were duty-cycled, with most buoys sampling at a rate of 18
minutes off and 2 minutes on. The SWFSC team analyzed the data and
detected echolocation pulses produced by beaked and sperm whales. These
were classified by peak frequency and spectral signature and identified
to the species level (see Simonis, Trickey, et al. (2020) for details on
this survey and Rand, Wood, and Oswald (2022) for duty-cycle study).
Soundscape metrics were binned by peak frequency into low, medium, high
frequency bands, along with the 1, 5, 10, 25, 75, 95, and 99 percentile
distribution from each metric.

\st{Soundscape metrics were rounded to the nearest 20 minutes (:00, :20,
:40) and converted to the UTC time zone. These metrics include
third-octave level (TOL) bands from 63 to 20,000 Hz and broadband
(20-24,000 Hz) metrics.} Several TOL levels were chosen as proxies to
vessel or sonar frequencies.

\st{Whale detection timestamps were rounded to nearest 20 minutes based
on start time to align with the soundscape metrics. A separate column
was created as a common join field that contained nearest timestamp in
UTC to that of buoy tracks (see below).} Two types of beaked whale
detections (species code \emph{?BW} and \emph{?BWC}) were removed from
analysis as the species was unknown.

**\[New addition as of 08/25/23 -** Two SPOT buoys sit on every buoy, so some GPS points were recorded twice. These duplicate tracks were eliminated.\]
\st{Buoy tracks were similarly rounded to nearest 20 minutes.} Seven
buoys (1,2,3,5,6,9,11) were lost at sea and two buoys had corrupted data
(4,17) so these were removed from the analysis. Buoys 14 and 15 were the
same and therefore consolidated to 14. These three preliminary datasets
were joined together based on a common buoy number (column ``station'')
and time field column ``UTC''). Whale detections (column ``Wpresence'')
were reduced to binary data for absence/presence and all other fields of
that dataset were eliminated.

\hypertarget{environmental-data}{%
\subsubsection{\texorpdfstring{\textbf{Environmental
Data}}{Environmental Data}}\label{environmental-data}}

Possible environmental and oceanographic predictors of cetacean habitat
were extracted from the NOAA Environmental Research Division Data Access
Program (ERDDAP) server
((\url{https://coastwatch.pfeg.noaa.gov/erddap/}) and Hybrid Coordinate
Ocean Model (HYCOM) using the \emph{matchEnvData} function in R package
\emph{PAMmisc}. Six modeled variables at depth (mixed layer depth, mixed
layer temperature, temperature at 400m, thermocline temperature,
thermocline depth, salinity at 400m) were calculated using a formula
from McCullough et al. (2021). Such variables are especially important
to incorporate as it better reflects the deep environment that beaked
and sperm whales are known to inhabit. Distance to slope and bathymetric
slope were calculated from a bathymetry TIF acquired from the General
Bathymetric Chart of the Oceans (GEBCO).

\[See Virgili et al 2019 for extracting slope from GEBCO Data and dynamic variables!!\]

\hypertarget{modeling}{%
\subsubsection{\texorpdfstring{\textbf{Modeling}}{Modeling}}\label{modeling}}

We used a Random Forest (RF) model in R (R Core Team, 2023) using the
package () to identify most important environmental or acoustic
predictors for beaked and sperm whales. A separate model was run for
each species.

We built generalized additive models in R using the package \emph{mcgv}
to relate beaked and sperm whale detections to an environmental or
acoustic variable.

\hypertarget{results}{%
\subsection{Results}\label{results}}

(Describe simple statistical relationships, correlations, exploratory
data findings (highest, lowest, etc., time of day ( Ziegenhorn et al.,
2023 ) , drift with most detections/day), where/when were most
detections heard, number of species)

\begin{itemize}
\item
  Buoy 18 had the most whale detections
\item
  ZC was the most frequently detected whale
\item
  The loudest sound levels were only occupied by PM
\item
  Density of ship encounters
\item
  Report on environmental conditions for ZC/PM?
\item
  Report on analysis of env variables WITH and WITHOUT soundscape
  metrics (see Fiedler, Becker, Forney, Barlow, and Moore (2023),
  results 3.1)
\end{itemize}

Weekly/daily plot of acoustic presence of whales and ships over time
from each drift? See JS Trickey et al.~2022, Fig 3 Which buoy had the
most AIS overlap and when?

Look at power spectral density? Haver et al. (2020)

The band at 125 Hertz is often associated with shipping vessel noise and
the 2-5 kHz range with mid-frequency naval sonar ( Garrett et al. (2016)
). Beaked whales are known to be sensitive to sonar in this latter range
(cite)

AIC explanation Symonds and Moussalli (2011)

\hypertarget{discussion}{%
\subsection{Discussion}\label{discussion}}

\[Laws and Regulations\]

While there is consensus that anthropogenic noise in the ocean is
increasing, policies that aim to mitigate chronic noise are generally
lacking. This is due to knowledge gaps regarding the long-term and
cumulative effects of noise on certain species (Markus and Sánchez
2018). International and regional efforts are underway (see Chou et
al.~(2021)) but are not fully complete. Within the European Union, the
Marine Strategy Framework Directive establishes a directive about
environmental regulation on marine noise, specifically requiring member
states to adopt strategies in compliance with threshold values for
impulsive and continuous noise (``Directive 2008/56/EC of the European
Parliament and of the Council of 17 June 2008 Establishing a Framework
for Community Action in the Field of Marine Environmental Policy (Marine
Strategy Framework Directive) (Text with EEA Relevance)'' 2008; Borsani,
J.F. et al.~2023). Some studies have independently tested these
threshold values (Sebastianutto et al.~2015; Garrett et al.~2016; Haver
et al.~2021) or used them as guidelines in their study (Mustonen et
al.~2019). The National Oceanic and Atmospheric Administration (NOAA)
has developed an Ocean Noise Strategy Roadmap, an agency-wise approach
to mitigating both chronic and acute effects of noise exposure to
animals (NOAA 2018). Within it, the authors agree upon impairment
thresholds for mid-frequency cetaceans (NOAA 2018). Because chronic
effects are not fully understood, they remain to be regulated. The
International Maritime Organization (IMO) has also been addressing noise
pollution from commercial vessels through quieter ship design
(International Maritime Organization 2014). Updates to their 2014
guidelines are set to be finalized by January 2024. Finally, Canada has
launched a Quiet Vessel Initiative and is expected to publish a Ocean
Noise Strategy for underwater noise management in 2023 (Breeze et
al.~2022).

\[Known limitations to this study\]\\
Duty cycle whale detection and soundscape method knowingly omits 18 of
20 minutes of data. A sperm whale's true location relative to buoy is
unknown so preferred habitat may not reflect the environmental variables
that are associated with buoy that detects it (Barkley et al.~2021)
Rounding tracks and soundscape metrics date/times to nearest 20 minutes
is not perfect. Often, times did not line up with :00, :20, :40 and were
either omitted or duplicated.

This study only captures half of the year. If BW/SW have seasonal
preferences outside of the study period, it was not captured here.

Are BW or SW known to have seasonal movements? Robbins et al. (2022)
suggest they do, though there's conflicting evidence.

Non-AIS vessels or small vessels (fishing, recreational) may contribute
to the soundscape in relevant frequency bands (Hermannsen et al. (2019)
, Hildebrand (2009) ) and this does not get captured here.

Does not account for the differences in relative noise contribution by
vessel type (as Hatch et al. (2008) does)

\[Add here.\]

This work will contribute to establishing baseline conditions of noise
exposure to inform wind farm development and shipping lane changes. In
better understanding this, human activities can be altered to avoid
overlap with sensitive species Hooker et al. (2019).

Both of these outcomes are of high importance to managers and agencies
especially NOAA and BOEM (see 2023-2024 report). Within their Ocean
Noise Strategy Roadmap, NOAA is specifically seeking 1) a quantification
of the spatial and temporal variability of ambient noise conditions, and
2) understanding of how anthropogenic sound sources contribute to the
soundscape. Co-occurrence of buoys and vessels will provide fine-scale
snapshots of sound levels surrounding vessels and concurrent whale
foraging activity. With AIS, we can calculate distance from buoy to
vessel and sound levels in those moments. And 3) NOAA is seeking more
information on vocally active species' distributions. This is part of a
bill introduced by Congress to ``assess underwater sound in
high-priority'' environments
(\url{https://www.congress.gov/bill/117th-congress/house-bill/6987}).

Outside of cetacean research, similar designs of drifting buoys have
been used in estuarine and coral reef soundscape studies Lillis et al.
(2018).

The year 2018 was characterized in the CCE with slightly warmer than
average sea surface temperatures as the recent El-Nino and marine
heatwave events were ending (Harvey et al.~2019).

Croll, Clark, Calambokidis, Ellison, and Tershy (2001) found that whale
presence more closely related to prey abundance than sound levels

Can use non-parametric Kruskal-Wallis one-way analysis of variance test
to examine \emph{if number of whale detections per hour differed
depending on presence/absence of vessels} (see Trickey et al.~2022).

\[Recommendations?\]

\hypertarget{acknowledgements}{%
\subsection{Acknowledgements}\label{acknowledgements}}

Taiki - help with code!\\
Megan McKenna - guidance on AIS questions.

Add here.

\hypertarget{tables}{%
\subsubsection{Tables}\label{tables}}

\begin{enumerate}
\def\labelenumi{\arabic{enumi})}
\tightlist
\item
  Species studied:\\
\end{enumerate}

\begin{itemize}
\tightlist
\item
  ZC Cuvier's beaked whale~\emph{Ziphius cavirostris (IUCN status: least
  concern)\\
  -} BW43, Perrin's beaked whale~\emph{M}.~\emph{perrini (IUCN status:
  endangered)} confirmed in 2021 Baumann-Pickering et al. (2013) Barlow
  et al. (2021)\\
\item
  BW37V, possibly Hubbs' beaked whale \emph{M. carlhubbsi (IUCN status:
  data deficient)} Simonis, Trickey, et al. (2020)\\
\item
  BB, -- Baird's beaked whale \emph{B. bairidii} \emph{(IUCN status:
  least concern)}\\
\item
  BWC, possibly ginkgo-toothed beaked whale \emph{M. ginkgodens (IUCN
  status: data deficient)} Simonis, Trickey, et al. (2020)\\
\item
  PM, sperm whale \emph{(IUCN status: vulnerable)}
\end{itemize}

\begin{enumerate}
\def\labelenumi{\arabic{enumi})}
\setcounter{enumi}{1}
\tightlist
\item
  Environmental covariates (group by physiographic, oceanographic, etc)
\end{enumerate}

\begin{longtable}[]{@{}
  >{\raggedright\arraybackslash}p{(\columnwidth - 6\tabcolsep) * \real{0.2361}}
  >{\raggedright\arraybackslash}p{(\columnwidth - 6\tabcolsep) * \real{0.2361}}
  >{\raggedright\arraybackslash}p{(\columnwidth - 6\tabcolsep) * \real{0.2917}}
  >{\raggedright\arraybackslash}p{(\columnwidth - 6\tabcolsep) * \real{0.2361}}@{}}
\toprule\noalign{}
\begin{minipage}[b]{\linewidth}\raggedright
Covariate
\end{minipage} & \begin{minipage}[b]{\linewidth}\raggedright
Resolution
\end{minipage} & \begin{minipage}[b]{\linewidth}\raggedright
Source
\end{minipage} & \begin{minipage}[b]{\linewidth}\raggedright
Justification
\end{minipage} \\
\midrule\noalign{}
\endhead
\bottomrule\noalign{}
\endlastfoot
& & & \\
Distance to shore (m) & 0.01-Degree Grid & ERDAPP & \\
Distance to continental slope (m) & 15 arc sec & Derived from General
Bathymetric Chart of the Oceans (GEBCO) Weatherall et al. (2015) &
Barlow et al. (2009) \\
Bathymetric slope (°) & 15 arc sec & Derived from GEBCO tif & Becker et
al. (2010) Virgili et al. (2022) \\
Depth (m) & 15 arc sec & ERDAPP & Forney et al. (2012) Becker et al.
(2010) Virgili et al. (2022) \\
Wind curl stress & 1°, 6-hourly & ERDDAP & Mannocci, Monestiez, Spitz,
and Ridoux (2015) \\
Mixed layer depth (m) & 0.08°, 3-hourly & HYCOM + NCODA, GLBy0.08/expt
93.0, calculated with variable representative isotherm method (Fiedler
2010), as described in McCullough et al 2021. & McCullough et al.
(2021), Forney et al. (2012) Moore (2021) \\
Thermocline temperature (°C) & 0.08°, 3-hourly & HYCOM + NCODA,
GLBy0.08/expt 93.0, calculated with variable representative isotherm
method (Fiedler 2010), as described in McCullough et al 2021. &
McCullough et al. (2021) \\
Thermocline depth (m) & 0.08°, 3-hourly & HYCOM + NCODA, GLBy0.08/expt
93.0, calculated with variable representative isotherm method (Fiedler
2010), as described in McCullough et al 2021. & \\
Mixed layer depth Temperature (°C) & 0.08°, 3-hourly & Calculated from
HYCOM & McCullough et al. (2021) \\
Salinity at 400m & 0.08°, 3-hourly & HYCOM + NCODA, GLBy0.08/expt 93.0
& \\
Temperature at 400m (°C) & 0.08°, 3-hourly & HYCOM + NCODA,
GLBy0.08/expt 93.0 & \\
SST mean & 0.01° , daily & ERDAPP & Becker 2007, Forney et al. (2012)
Virgili et al. (2022) \\
CHL & 4 km, monthly & ERDAPP & Forney et al. (2012) Virgili et al.
(2019) \\
Sea Surface Height Anomalies (m) & daily & ERDAPP & Virgili et al.
(2019) \\
\end{longtable}

Can show table of variables and justification by papers (Virgili et
al.~2022 and Virgili et al 2019)

Forney et al. (2012) shows how BW model had a lot of uncertainty
regarding distribution predictors. So few detections that some spp had
to be lumped together (also Barlow et al. (2009)).

\hypertarget{refs}{}
\begin{CSLReferences}{1}{0}
\leavevmode\vadjust pre{\hypertarget{ref-affatati2022}{}}%
Affatati, A., Scaini, C., \& Salon, S. (2022). Ocean Sound Propagation
in a Changing Climate: Global Sound Speed Changes and Identification of
Acoustic Hotspots. \emph{Earth's Future}, \emph{10}(3), e2021EF002099.
\url{https://doi.org/10.1029/2021EF002099}

\leavevmode\vadjust pre{\hypertarget{ref-aguilardesoto2017}{}}%
Aguilar de Soto, N. A. de, Martín, V., Silva, M., Edler, R., Reyes, C.,
Carrillo, M., \ldots{} Carroll, E. (2017). True{'}s beaked whale
(Mesoplodon mirus) in Macaronesia. \emph{PeerJ}, \emph{5}, e3059.
\url{https://doi.org/10.7717/peerj.3059}

\leavevmode\vadjust pre{\hypertarget{ref-aguilarsoto2006}{}}%
Aguilar Soto, N., Johnson, M., Madsen, P. T., Tyack, P. L., Bocconcelli,
A., \& Fabrizio Borsani, J. (2006). Does Intense Ship Noise Disrupt
Foraging in Deep-Diving Cuvier's Beaked Whales (ziphius Cavirostris)?
\emph{Marine Mammal Science}, \emph{22}(3), 690--699.
\url{https://doi.org/10.1111/j.1748-7692.2006.00044.x}

\leavevmode\vadjust pre{\hypertarget{ref-amano2003}{}}%
Amano, M., \& Yoshioka, M. (2003). Sperm whale diving behavior monitored
using a suction-cup-attached TDR tag. \emph{Marine Ecology Progress
Series}, \emph{258}, 291--295. Retrieved from
\url{https://www.jstor.org/stable/24867054}

\leavevmode\vadjust pre{\hypertarget{ref-andruxe92011}{}}%
André, M., Schaar, M. van der, Zaugg, S., Houégnigan, L., Sánchez, A.
M., \& Castell, J. V. (2011). Listening to the Deep: Live monitoring of
ocean noise and cetacean acoustic signals. \emph{Marine Pollution
Bulletin}, \emph{63}(1), 18--26.
\url{https://doi.org/10.1016/j.marpolbul.2011.04.038}

\leavevmode\vadjust pre{\hypertarget{ref-arranz2023}{}}%
Arranz, P., Miranda, D., Gkikopoulou, K. C., Cardona, A., Alcazar, J.,
Aguilar de Soto, N., \ldots{} Marques, T. A. (2023). Comparison of
visual and passive acoustic estimates of beaked whale density off el
hierro, canary islands. \emph{The Journal of the Acoustical Society of
America}, \emph{153}(4), 2469. \url{https://doi.org/10.1121/10.0017921}

\leavevmode\vadjust pre{\hypertarget{ref-awbery2022}{}}%
Awbery, T. (2022). \emph{Spatial distribution and encounter rates of
delphinids and deep diving cetaceans in the eastern mediterranean sea of
turkey and the extent of overlap with areas of dense marine traffic}.
Retrieved from
\url{https://www.frontiersin.org/articles/10.3389/fmars.2022.860242/full}

\leavevmode\vadjust pre{\hypertarget{ref-baird2008}{}}%
Baird, R. W. (2008). Diel variation in beaked whale diving behavior.
\emph{Marine Mammal Science}.
\url{https://doi.org/10.1111/j.1748-7692.2008.00211.x}

\leavevmode\vadjust pre{\hypertarget{ref-baird2006}{}}%
Baird, R. W., Webster, D. L., McSweeney, D. J., Ligon, A. D., Schorr, G.
S., \& Barlow, J. (2006). Diving behaviour of cuvier{'}s (ziphius
cavirostris) and blainville{'}s (mesoplodon densirostris) beaked whales
in hawai{`}i. \emph{Canadian Journal of Zoology}, \emph{84}(8),
1120--1128. \url{https://doi.org/10.1139/Z06-095}

\leavevmode\vadjust pre{\hypertarget{ref-barlow2021}{}}%
Barlow, J., Cárdenas-Hinojosa, G., Henderson, E. E., Breese, D.,
Lopez-Arzate, D., Hidalgo-Pla, E., \& Taylor, B. N. (2021). Unique
morphological and acoustic characteristics of beaked whales (mesoplodon
sp.) off the west coast of baja california, mexico. \emph{Marine Mammal
Science}. \url{https://doi.org/10.1111/mms.12853}

\leavevmode\vadjust pre{\hypertarget{ref-barlow2009}{}}%
Barlow, J., Ferguson, M. C., Becker, E. A., Redfern, J. V., Forney, K.
A., Vilchis, I. L., \ldots{} Ballance, L. T. (2009). \emph{Predictive
modeling of cetacean densities in the eastern pacific ocean}. Retrieved
from
\url{https://repository.library.noaa.gov/view/noaa/3666/noaa_3666_DS1.pdf}

\leavevmode\vadjust pre{\hypertarget{ref-barlow2007}{}}%
Barlow, J., \& Forney, K. A. (2007). Abundance and population density of
cetaceans in the california current ecosystem. \emph{Fishery Bulletin},
\emph{105}(4), 509--527. Retrieved from
\url{https://go.gale.com/ps/i.do?p=AONE\&sw=w\&issn=00900656\&v=2.1\&it=r\&id=GALE\%7CA203535836\&sid=googleScholar\&linkaccess=abs}

\leavevmode\vadjust pre{\hypertarget{ref-barlow2005}{}}%
Barlow, J., \& Gisiner, R. (2005). Mitigating, monitoring and assessing
the effects of anthropogenic sound on beaked whales. \emph{J. Cetacean
Res. Manage.}, \emph{7}(3), 239--249.
\url{https://doi.org/10.47536/jcrm.v7i3.734}

\leavevmode\vadjust pre{\hypertarget{ref-baumann-pickering2013}{}}%
Baumann-Pickering, S., McDonald, M. A., Simonis, A. E., Solsona Berga,
A., Merkens, K. P. B., Oleson, E. M., \ldots{} Hildebrand, J. A. (2013).
Species-specific beaked whale echolocation signals. \emph{The Journal of
the Acoustical Society of America}, \emph{134}(3), 2293--2301.
\url{https://doi.org/10.1121/1.4817832}

\leavevmode\vadjust pre{\hypertarget{ref-baumann-pickering2014}{}}%
Baumann-Pickering, S., Roch, M. A., Jr, R. L. B., Simonis, A. E.,
McDonald, M. A., Solsona-Berga, A., \ldots{} Hildebrand, J. A. (2014).
Spatio-Temporal Patterns of Beaked Whale Echolocation Signals in the
North Pacific. \emph{PLOS ONE}, \emph{9}(1), e86072.
\url{https://doi.org/10.1371/journal.pone.0086072}

\leavevmode\vadjust pre{\hypertarget{ref-becker2010}{}}%
Becker, E. A., Forney, K. A., Ferguson, M. C., Foley, D. G., Smith, R.
C., Barlow, J., \& Redfern, J. V. (2010). Comparing california current
cetacean{\textendash}habitat models developed using in situ and remotely
sensed sea surface temperature data. \emph{Marine Ecology Progress
Series}, \emph{413}, 163--183. Retrieved from
\url{https://www.jstor.org/stable/24875187}

\leavevmode\vadjust pre{\hypertarget{ref-bernaldodequiros2019}{}}%
Bernaldo de Quiros, Y., Fernández, A., Baird, R. W., Brownell, R. L.,
Aguilar de Soto, N., Allen, D., \ldots{} Schorr, G. (2019). Advances in
research on the impacts of anti-submarine sonar on beaked whales.
\emph{Proceedings of the Royal Society B: Biological Sciences},
\emph{286}(1895), 1--9. Retrieved from
\url{https://royalsocietypublishing.org/doi/full/10.1098/rspb.2018.2533}

\leavevmode\vadjust pre{\hypertarget{ref-borsanij.f.2023}{}}%
Borsani, J.F., Juretzek C., Klauson A., Leaper R., Le Courtois F.,
Liebschner A., Maglio A., Mueller A. , Norro A., Novellino A., Outinen
O., Popit A., Prospathopoulos A., Sigray P., Thomsen F., Tougaard J.,
Vukadin P., and Weilgart L., Borsani, J.F., Andersson M., André M.,
Azzellino A., Bou M., Castellote M., Ceyrac L., Dellong D., Folegot T.,
Hedgeland D., \& Juretzek C., Klauson A., Leaper R., Le Courtois F.,
Liebschner A., Maglio A., Mueller A. , Norro A., Novellino A., Outinen
O.,. (2023). \emph{Setting EU threshold values for continuous underwater
sound, technical group on underwater noise (TG NOISE)}.

\leavevmode\vadjust pre{\hypertarget{ref-breeze2022}{}}%
Breeze, H., Nolet, V., Thomson, D., Wright, A. J., Marotte, E., \&
Sanders, M. (2022). Efforts to advance underwater noise management in
Canada: Introduction to the Marine Pollution Bulletin Special Issue.
\emph{Marine Pollution Bulletin}, \emph{178}, 113596.
\url{https://doi.org/10.1016/j.marpolbul.2022.113596}

\leavevmode\vadjust pre{\hypertarget{ref-buxton2017}{}}%
Buxton, R. T., McKenna, M. F., Mennitt, D., Fristrup, K., Crooks, K.,
Angeloni, L., \& Wittemyer, G. (2017). Noise pollution is pervasive in
u.s. Protected areas. \emph{Science}, \emph{356}(6337), 531--533.
\url{https://doi.org/10.1126/science.aah4783}

\leavevmode\vadjust pre{\hypertarget{ref-chahouri2022}{}}%
Chahouri, A., Elouahmani, N., \& Ouchene, H. (2022). Recent progress in
marine noise pollution: A thorough review. \emph{Chemosphere},
\emph{291}, 132983.
\url{https://doi.org/10.1016/j.chemosphere.2021.132983}

\leavevmode\vadjust pre{\hypertarget{ref-cholewiak2017}{}}%
Cholewiak, D., DeAngelis, A. I., Palka, D., Corkeron, P. J., \& Van
Parijs, S. M. (2017). Beaked whales demonstrate a marked acoustic
response to the use of shipboard echosounders. \emph{Royal Society Open
Science}, \emph{4}(12), 170940.
\url{https://doi.org/10.1098/rsos.170940}

\leavevmode\vadjust pre{\hypertarget{ref-chou2021}{}}%
Chou, E., Southall, B. L., Robards, M., \& Rosenbaum, H. C. (2021).
International policy, recommendations, actions and mitigation efforts of
anthropogenic underwater noise. \emph{Ocean \& Coastal Management},
\emph{202}, 105427.
\url{https://doi.org/10.1016/j.ocecoaman.2020.105427}

\leavevmode\vadjust pre{\hypertarget{ref-coleman2008}{}}%
Coleman, K. (2008). Research review of collaborative ecosystem-based
management in the california current large marine ecosystem.
\emph{Coastal Management}, \emph{36}(5), 484--494.
\url{https://doi.org/10.1080/08920750802395541}

\leavevmode\vadjust pre{\hypertarget{ref-cooperman2022}{}}%
Cooperman, A., Duffy, P., Hall, M., Lozon, E., Shields, M., \& Musail,
W. (2022). \emph{Assessment of offshore wind energy leasing areas for
humboldt and morro bay wind energy areas, california}. Retrieved from
\url{https://www.nrel.gov/docs/fy22osti/82341.pdf}

\leavevmode\vadjust pre{\hypertarget{ref-cox2006}{}}%
Cox, T. M., Ragen, T. J., Read, A. J., Vos, E., Baird, R. W., Balcomb,
K., \& Benner, L. (2006). Understanding the impacts of anthropogenic
sound on beaked whales. \emph{J. Cetacean Res. Manage}.

\leavevmode\vadjust pre{\hypertarget{ref-croll2001}{}}%
Croll, D. A., Clark, C. W., Calambokidis, J., Ellison, W. T., \& Tershy,
B. R. (2001). Effect of anthropogenic low-frequency noise on the
foraging ecology of Balaenoptera whales. \emph{Animal Conservation},
\emph{4}(1), 13--27. \url{https://doi.org/10.1017/S1367943001001020}

\leavevmode\vadjust pre{\hypertarget{ref-curuxe92013}{}}%
Curé, C., Antunes, R., Alves, A. C., Visser, F., Kvadsheim, P. H., \&
Miller, P. J. O. (2013). Responses of male sperm whales (Physeter
macrocephalus) to killer whale sounds: implications for anti-predator
strategies. \emph{Scientific Reports}, \emph{3}(1), 1579.
\url{https://doi.org/10.1038/srep01579}

\leavevmode\vadjust pre{\hypertarget{ref-deruiter2013}{}}%
DeRuiter, S. L., Southall, B. L., Calambokidis, J., Zimmer, W. M. X.,
Sadykova, D., Falcone, E. A., \ldots{} Tyack, P. L. (2013). First direct
measurements of behavioural responses by cuvier's beaked whales to
mid-frequency active sonar. \emph{Biology Letters}, \emph{9}(4),
20130223. \url{https://doi.org/10.1098/rsbl.2013.0223}

\leavevmode\vadjust pre{\hypertarget{ref-directiv2008}{}}%
\emph{Directive 2008/56/EC of the European Parliament and of the Council
of 17 June 2008 establishing a framework for community action in the
field of marine environmental policy (Marine Strategy Framework
Directive) (Text with EEA relevance)}. (2008). Retrieved from
\url{http://data.europa.eu/eli/dir/2008/56/oj/eng}

\leavevmode\vadjust pre{\hypertarget{ref-duarte2021}{}}%
Duarte, C. M., Chapuis, L., Collin, S. P., Costa, D. P., Devassy, R. P.,
Eguiluz, V. M., \ldots{} Juanes, F. (2021). The soundscape of the
anthropocene ocean. \emph{Science}, \emph{371}(6529), eaba4658.
\url{https://doi.org/10.1126/science.aba4658}

\leavevmode\vadjust pre{\hypertarget{ref-erbe2014}{}}%
Erbe, C., Williams, R., Sandilands, D., \& Ashe, E. (2014). Identifying
Modeled Ship Noise Hotspots for Marine Mammals of Canada's Pacific
Region. \emph{PLOS ONE}, \emph{9}(3), e89820.
\url{https://doi.org/10.1371/journal.pone.0089820}

\leavevmode\vadjust pre{\hypertarget{ref-fahlman2014}{}}%
Fahlman, A., Tyack, P., Miller, P., \& Kvadsheim, P. (2014). How
man-made interference might cause gas bubble emboli in deep diving
whales. \emph{Frontiers in Physiology}, \emph{5}. Retrieved from
\url{https://www.frontiersin.org/articles/10.3389/fphys.2014.00013}

\leavevmode\vadjust pre{\hypertarget{ref-fais2015}{}}%
Fais, A., Aguilar Soto, N., Johnson, M., Pérez-González, C., Miller, P.
J. O., \& Madsen, P. T. (2015). Sperm whale echolocation behaviour
reveals a directed, prior-based search strategy informed by prey
distribution. \emph{Behavioral Ecology and Sociobiology}, \emph{69}(4),
663--674. \url{https://doi.org/10.1007/s00265-015-1877-1}

\leavevmode\vadjust pre{\hypertarget{ref-fais2016}{}}%
Fais, A., Johnson, M., Wilson, M., Aguilar Soto, N., \& Madsen, P. T.
(2016). Sperm whale predator-prey interactions involve chasing and
buzzing, but no acoustic stunning. \emph{Scientific Reports},
\emph{6}(1), 28562. \url{https://doi.org/10.1038/srep28562}

\leavevmode\vadjust pre{\hypertarget{ref-falcone2017}{}}%
Falcone, E. A., Schorr, G. S., Watwood, S. L., DeRuiter, S. L., Zerbini,
A. N., Andrews, R. D., \ldots{} Moretti, D. J. (2017). Diving behaviour
of cuvier's beaked whales exposed to two types of military sonar.
\emph{Royal Society Open Science}, \emph{4}(8), 170629.
\url{https://doi.org/10.1098/rsos.170629}

\leavevmode\vadjust pre{\hypertarget{ref-federalregister2003}{}}%
Federal Register. (2003). \emph{Automatic identification system; vessel
carriage requirement}. Retrieved from
\url{https://www.federalregister.gov/documents/2003/07/01/03-16191/automatic-identification-system-vessel-carriage-requirement}

\leavevmode\vadjust pre{\hypertarget{ref-fernuxe1ndez2005}{}}%
Fernández, A., Edwards, J. F., Rodríguez, F., Monteros, A. E. de los,
Herráez, P., Castro, P., \ldots{} Arbelo, M. (2005). {``}Gas and Fat
Embolic Syndrome{''} Involving a Mass Stranding of Beaked Whales (Family
Ziphiidae) Exposed to Anthropogenic Sonar Signals. \emph{Veterinary
Pathology}, \emph{42}(4), 446--457.
\url{https://doi.org/10.1354/vp.42-4-446}

\leavevmode\vadjust pre{\hypertarget{ref-fiedler2023}{}}%
Fiedler, P. C., Becker, E. A., Forney, K. A., Barlow, J., \& Moore, J.
E. (2023). Species distribution modeling of deep-diving cetaceans.
\emph{Marine Mammal Science}, \emph{39}(4), 1178--1203.
\url{https://doi.org/10.1111/mms.13057}

\leavevmode\vadjust pre{\hypertarget{ref-field2006}{}}%
Field, J. C., \& Francis, R. C. (2006). Considering ecosystem-based
fisheries management in the california current. \emph{Marine Policy},
\emph{30}(5), 552--569.
\url{https://doi.org/10.1016/j.marpol.2005.07.004}

\leavevmode\vadjust pre{\hypertarget{ref-fleishman2023}{}}%
Fleishman, E., Cholewiak, D., Gillespie, D., Heble, T., Klinck, H.,
Nosal, E.-M., \& Roch, M. A. (2023). Ecological inferences about marine
mammals from passive acoustic data. \emph{Biological Reviews}.
https://doi.org/\url{https://doi.org/10.1111/brv.12969}

\leavevmode\vadjust pre{\hypertarget{ref-forney2012}{}}%
Forney, K. A., Ferguson, M. C., Becker, E. A., Fiedler, P. C., Redfern,
J. V., Barlow, J., \ldots{} Ballance, L. T. (2012). Habitat-based
spatial models of cetacean density in the eastern Pacific Ocean.
\emph{Endangered Species Research}, \emph{16}(2), 113--133.
\url{https://doi.org/10.3354/esr00393}

\leavevmode\vadjust pre{\hypertarget{ref-francis2012}{}}%
Francis, C. D., Kleist, N. J., Ortega, C. P., \& Cruz, A. (2012). Noise
pollution alters ecological services: Enhanced pollination and disrupted
seed dispersal. \emph{Proceedings of the Royal Society B: Biological
Sciences}, \emph{279}(1739), 2727--2735.
\url{https://doi.org/10.1098/rspb.2012.0230}

\leavevmode\vadjust pre{\hypertarget{ref-frisk2012}{}}%
Frisk, G. V. (2012). Noiseonomics: The relationship between ambient
noise levels in the sea and global economic trends. \emph{Scientific
Reports}, \emph{2}(1), 437. \url{https://doi.org/10.1038/srep00437}

\leavevmode\vadjust pre{\hypertarget{ref-garrett2016}{}}%
Garrett, J. K., Blondel, Ph., Godley, B. J., Pikesley, S. K., Witt, M.
J., \& Johanning, L. (2016). Long-term underwater sound measurements in
the shipping noise indicator bands 63Hz and 125Hz from the port of
Falmouth Bay, UK. \emph{Marine Pollution Bulletin}, \emph{110}(1),
438--448. \url{https://doi.org/10.1016/j.marpolbul.2016.06.021}

\leavevmode\vadjust pre{\hypertarget{ref-greig2020}{}}%
Greig, N. C., Hines, E. M., Cope, S., \& Liu, X. (2020). Using satellite
AIS to analyze vessel speeds off the coast of washington state, u.s., as
a risk analysis for cetacean-vessel collisions. \emph{Frontiers in
Marine Science}, \emph{7}. Retrieved from
\url{https://www.frontiersin.org/articles/10.3389/fmars.2020.00109}

\leavevmode\vadjust pre{\hypertarget{ref-harvey}{}}%
Harvey, C. J., Garfield, N., Williams, G. D. (Gregory. D., Tolimieri,
N., Schroeder, I., Andrews, K. S., \ldots{} Juhasz, C. (n.d.).
\emph{Ecosystem status report of the california current for 2019: A
summary of ecosystem indicators compiled by the california current
integrated ecosystem assessment team (CCEIA)}.
\url{https://doi.org/10.25923/P0ED-KE21}

\leavevmode\vadjust pre{\hypertarget{ref-hatch2008}{}}%
Hatch, L., Clark, C., Merrick, R., Van Parijs, S., Ponirakis, D.,
Schwehr, K., \ldots{} Wiley, D. (2008). Characterizing the Relative
Contributions of Large Vessels to Total Ocean Noise Fields: A Case Study
Using the Gerry E. Studds Stellwagen Bank National Marine Sanctuary.
\emph{Environmental Management}, \emph{42}(5), 735--752.
\url{https://doi.org/10.1007/s00267-008-9169-4}

\leavevmode\vadjust pre{\hypertarget{ref-haver2021}{}}%
Haver, S. M., Adams, J. D., Hatch, L. T., Van Parijs, S. M., Dziak, R.
P., Haxel, J., \ldots{} Gedamke, J. (2021). Large vessel activity and
low-frequency underwater sound benchmarks in united states waters.
\emph{Frontiers in Marine Science}, \emph{8}. Retrieved from
\url{https://www.frontiersin.org/articles/10.3389/fmars.2021.669528}

\leavevmode\vadjust pre{\hypertarget{ref-haver2019}{}}%
Haver, S. M., Fournet, M. E. H., Dziak, R. P., Gabriele, C., Gedamke,
J., Hatch, L. T., \ldots{} Van Parijs, S. M. (2019). Comparing the
underwater soundscapes of four u.s. National parks and marine
sanctuaries. \emph{Frontiers in Marine Science}, \emph{6}. Retrieved
from \url{https://www.frontiersin.org/articles/10.3389/fmars.2019.00500}

\leavevmode\vadjust pre{\hypertarget{ref-haver2020}{}}%
Haver, S. M., Rand, Z., Hatch, L. T., Lipski, D., Dziak, R. P., Gedamke,
J., \ldots{} Van Parijs, S. M. (2020). Seasonal trends and primary
contributors to the low-frequency soundscape of the cordell bank
national marine sanctuarya). \emph{The Journal of the Acoustical Society
of America}, \emph{148}(2), 845--858.
\url{https://doi.org/10.1121/10.0001726}

\leavevmode\vadjust pre{\hypertarget{ref-hawkins2017}{}}%
Hawkins, A. D., \& Popper, A. N. (2017). A sound approach to assessing
the impact of underwater noise on marine fishes and invertebrates.
\emph{ICES Journal of Marine Science}, \emph{74}(3), 635--651.
\url{https://doi.org/10.1093/icesjms/fsw205}

\leavevmode\vadjust pre{\hypertarget{ref-hermannsen2019}{}}%
Hermannsen, L., Mikkelsen, L., Tougaard, J., Beedholm, K., Johnson, M.,
\& Madsen, P. T. (2019). Recreational vessels without automatic
identification system (AIS) dominate anthropogenic noise contributions
to a shallow water soundscape. \emph{Scientific Reports}, \emph{9},
15477. \url{https://doi.org/10.1038/s41598-019-51222-9}

\leavevmode\vadjust pre{\hypertarget{ref-hildebrand2009}{}}%
Hildebrand, J. A. (2009). Anthropogenic and natural sources of ambient
noise in the ocean. \emph{Marine Ecology Progress Series}, \emph{395},
5--20. \url{https://doi.org/10.3354/meps08353}

\leavevmode\vadjust pre{\hypertarget{ref-hildebrand2015}{}}%
Hildebrand, J. A., Baumann-Pickering, S., Frasier, K. E., Trickey, J.
S., Merkens, K. P., Wiggins, S. M., \ldots{} Thomas, L. (2015). Passive
acoustic monitoring of beaked whale densities in the Gulf of Mexico.
\emph{Scientific Reports}, \emph{5}(1), 16343.
\url{https://doi.org/10.1038/srep16343}

\leavevmode\vadjust pre{\hypertarget{ref-hooker2019}{}}%
Hooker, S. K., De Soto, N. A., Baird, R. W., Carroll, E. L., Claridge,
D., Feyrer, L., \ldots{} Whitehead, H. (2019). Future directions in
research on beaked whales. \emph{Frontiers in Marine Science}, \emph{5}.
Retrieved from
\url{https://www.frontiersin.org/articles/10.3389/fmars.2018.00514}

\leavevmode\vadjust pre{\hypertarget{ref-internationalmaritimeorganization2014}{}}%
International Maritime Organization. (2014). \emph{GUIDELINES FOR THE
REDUCTION OF UNDERWATER NOISE FROM COMMERCIAL SHIPPING TO ADDRESS
ADVERSE IMPACTS ON MARINE LIFE}.

\leavevmode\vadjust pre{\hypertarget{ref-isojunno2016}{}}%
Isojunno, S., Curé, C., Kvadsheim, P. H., Lam, F.-P. A., Tyack, P. L.,
Wensveen, P. J., \& Miller, P. J. O. (2016). Sperm whales reduce
foraging effort during exposure to 1{\textendash}2 kHz sonar and killer
whale sounds. \emph{Ecological Applications}, \emph{26}(1), 77--93.
\url{https://doi.org/10.1890/15-0040}

\leavevmode\vadjust pre{\hypertarget{ref-isojunno2022}{}}%
Isojunno, S., von Benda-Beckmann, A. M., Wensveen, P. J., Kvadsheim, P.
H., Lam, F.-P. A., Gkikopoulou, K. C., \ldots{} Miller, P. J. O. (2022).
Sperm whales exhibit variation in echolocation tactics with depth and
sea state but not naval sonar exposures. \emph{Marine Mammal Science},
\emph{38}(2), 682--704. \url{https://doi.org/10.1111/mms.12890}

\leavevmode\vadjust pre{\hypertarget{ref-jarvis2022}{}}%
Jarvis, S. M., DiMarzio, N., Watwood, S., Dolan, K., \& Morrissey, R.
(2022). Automated detection and classification of beaked whale buzzes on
bottom-mounted hydrophones. \emph{Frontiers in Remote Sensing},
\emph{3}. Retrieved from
\url{https://www.frontiersin.org/articles/10.3389/frsen.2022.941838}

\leavevmode\vadjust pre{\hypertarget{ref-johnson2009}{}}%
Johnson, M., Soto, N. A. de, \& Madsen, P. T. (2009). Studying the
behaviour and sensory ecology of marine mammals using acoustic recording
tags: A review. \emph{Marine Ecology Progress Series}, \emph{395},
55--73. Retrieved from \url{https://www.jstor.org/stable/24874241}

\leavevmode\vadjust pre{\hypertarget{ref-joyce2020}{}}%
Joyce, T. W., Durban, J. W., Claridge, D. E., Dunn, C. A., Hickmott, L.
S., Fearnbach, H., \ldots{} Moretti, D. (2020). Behavioral responses of
satellite tracked Blainville's beaked whales (Mesoplodon densirostris)
to mid-frequency active sonar. \emph{Marine Mammal Science},
\emph{36}(1), 29--46. \url{https://doi.org/10.1111/mms.12624}

\leavevmode\vadjust pre{\hypertarget{ref-kunc2016}{}}%
Kunc, H. P., McLaughlin, K. E., \& Schmidt, R. (2016). Aquatic noise
pollution: Implications for individuals, populations, and ecosystems.
\emph{Proceedings of the Royal Society B: Biological Sciences},
\emph{283}(1836), 20160839. \url{https://doi.org/10.1098/rspb.2016.0839}

\leavevmode\vadjust pre{\hypertarget{ref-kvadsheim2012}{}}%
Kvadsheim, P., Miller, P., Tyack, P., Sivle, L., Lam, F.-P., \& Fahlman,
A. (2012). Estimated tissue and blood N2 levels and risk of in vivo
bubble formation in deep-, intermediate- and shallow diving toothed
whales during exposure to naval sonar. \emph{Frontiers in Physiology},
\emph{3}. Retrieved from
\url{https://www.frontiersin.org/articles/10.3389/fphys.2012.00125}

\leavevmode\vadjust pre{\hypertarget{ref-lillis2018}{}}%
Lillis, A., Caruso, F., Mooney, T. A., Llopiz, J., Bohnenstiehl, D., \&
Eggleston, D. B. (2018). Drifting hydrophones as an ecologically
meaningful approach to underwater soundscape measurement in coastal
benthic habitats. \emph{JEA}, \emph{2}(1), 1--1.
\url{https://doi.org/10.22261/JEA.STBDH1}

\leavevmode\vadjust pre{\hypertarget{ref-madsen2002}{}}%
Madsen, P. T., Mohl, B., Nielsen, B. K., \& Wahlberg, M. (2002). Male
sperm whale behaviour during exposures to distant seismic survey pulses.
\emph{Aquatic Mammals}.

\leavevmode\vadjust pre{\hypertarget{ref-madsen2006}{}}%
Madsen, P. T., Wahlberg, M., Tougaard, J., Lucke, K., \& Tyack, P.
(2006). Wind turbine underwater noise and marine mammals: Implications
of current knowledge and data needs. \emph{Marine Ecology Progress
Series}, \emph{309}, 279--295. Retrieved from
\url{https://www.jstor.org/stable/24869996}

\leavevmode\vadjust pre{\hypertarget{ref-mannocci2015}{}}%
Mannocci, L., Monestiez, P., Spitz, J., \& Ridoux, V. (2015).
Extrapolating cetacean densities beyond surveyed regions: habitat-based
predictions in the circumtropical belt. \emph{Journal of Biogeography},
\emph{42}(7), 1267--1280. \url{https://doi.org/10.1111/jbi.12530}

\leavevmode\vadjust pre{\hypertarget{ref-manzano-roth2023}{}}%
Manzano-Roth, R., Henderson, E. E., Alongi, G. C., Martin, C. R.,
Martin, S. W., \& Matsuyama, B. (2023). Dive characteristics of Cross
Seamount beaked whales from long-term passive acoustic monitoring at the
Pacific Missile Range Facility, Kauaʻi. \emph{Marine Mammal Science},
\emph{39}(1), 22--41. \url{https://doi.org/10.1111/mms.12959}

\leavevmode\vadjust pre{\hypertarget{ref-marcoux2006}{}}%
Marcoux, M., Whitehead, H., \& Rendell, L. (2006). Coda vocalizations
recorded in breeding areas are almost entirely produced by mature female
sperm whales (physeter macrocephalus). \emph{Canadian Journal of
Zoology}, \emph{84}(4), 609--614. \url{https://doi.org/10.1139/z06-035}

\leavevmode\vadjust pre{\hypertarget{ref-markus2018}{}}%
Markus, T., \& Sánchez, P. P. S. (2018). \emph{Managing and Regulating
Underwater Noise Pollution} (M. Salomon and T. Markus, Eds.).
\url{https://doi.org/10.1007/978-3-319-60156-4_52}

\leavevmode\vadjust pre{\hypertarget{ref-marques2012}{}}%
Marques, T. A., Thomas, L., Martin, S. W., Mellinger, D. K., Jarvis, S.,
Morrissey, R. P., \ldots{} DiMarzio, N. (2012). Spatially explicit
capture{\textendash}recapture methods to estimate minke whale density
from data collected at bottom-mounted hydrophones. \emph{Journal of
Ornithology}, \emph{152}(2), 445--455.
\url{https://doi.org/10.1007/s10336-010-0535-7}

\leavevmode\vadjust pre{\hypertarget{ref-mccullough2021}{}}%
McCullough, J. L. K., Wren, J. L. K., Oleson, E. M., Allen, A. N.,
Siders, Z. A., \& Norris, E. S. (2021). An acoustic survey of beaked
whales and kogia spp. In the mariana archipelago using drifting
recorders. \emph{Frontiers in Marine Science}, \emph{8}. Retrieved from
\url{https://www.frontiersin.org/articles/10.3389/fmars.2021.664292}

\leavevmode\vadjust pre{\hypertarget{ref-mellinger2007}{}}%
Mellinger, D. K., Stafford, K. M., Moore, S. E., Dziak, R. P., \&
Matsumoto, H. (2007). An overview of fixed passive acoustic observation
methods for cetaceans. \emph{Oceanography}, \emph{20}(4), 3645.
https://doi.org/\url{https://doi.org/10.5670/oceanog.2007.03}

\leavevmode\vadjust pre{\hypertarget{ref-miksis-olds2018}{}}%
Miksis-Olds, J. L., Martin, B., \& Tyack, P. L. (2018). Exploring the
ocean through soundscapes. \emph{Acoustics Today}, \emph{14}(1), 26--34.

\leavevmode\vadjust pre{\hypertarget{ref-miller2022}{}}%
Miller, Patrick J. O., Isojunno, S., Siegal, E., Lam, F.-P. A.,
Kvadsheim, P. H., \& Curé, C. (2022). Behavioral responses to predatory
sounds predict sensitivity of cetaceans to anthropogenic noise within a
soundscape of fear. \emph{Proceedings of the National Academy of
Sciences}, \emph{119}(13), e2114932119.
\url{https://doi.org/10.1073/pnas.2114932119}

\leavevmode\vadjust pre{\hypertarget{ref-miller2009}{}}%
Miller, P. J. O., Johnson, M. P., Madsen, P. T., Biassoni, N., Quero,
M., \& Tyack, P. L. (2009). Using at-sea experiments to study the
effects of airguns on the foraging behavior of sperm whales in the Gulf
of Mexico. \emph{Deep Sea Research Part I: Oceanographic Research
Papers}, \emph{56}(7), 1168--1181.
\url{https://doi.org/10.1016/j.dsr.2009.02.008}

\leavevmode\vadjust pre{\hypertarget{ref-moore2021}{}}%
Moore, J. E. (2021). \emph{Final report of the california current
ecosystem survey (CCES) 2018: A PacMAPPS study} (p. 187). Camarillo
(CA): US Department of the Interior, Bureau of Ocean Energy Management.

\leavevmode\vadjust pre{\hypertarget{ref-mustonen2019}{}}%
Mustonen, M., Klauson, A., Andersson, M., Clorennec, D., Folegot, T.,
Koza, R., \ldots{} Sigray, P. (2019). Spatial and Temporal Variability
of Ambient Underwater Sound in the Baltic Sea. \emph{Scientific
Reports}, \emph{9}(1), 13237.
\url{https://doi.org/10.1038/s41598-019-48891-x}

\leavevmode\vadjust pre{\hypertarget{ref-new2013}{}}%
New, L. F., Moretti, D. J., Hooker, S. K., Costa, D. P., \& Simmons, S.
E. (2013). Using Energetic Models to Investigate the Survival and
Reproduction of Beaked Whales (family Ziphiidae). \emph{PLOS ONE},
\emph{8}(7), e68725. \url{https://doi.org/10.1371/journal.pone.0068725}

\leavevmode\vadjust pre{\hypertarget{ref-noaa2018}{}}%
NOAA. (2018). \emph{2018 revisions to: Technical guidance for assessing
the effects of anthropogenic sound on marine mammal hearing (version
2.0): Underwater thresholds for onset of permanent and temporary
threshold shifts}.

\leavevmode\vadjust pre{\hypertarget{ref-noaafisheries}{}}%
NOAA Fisheries, N. (n.d.). \emph{Species directory \textbar{} NOAA
fisheries}. Retrieved from
\url{https://www.fisheries.noaa.gov/species-directory?oq=\&field_species_categories_vocab=54\&field_region_vocab=1000001126\&items_per_page=350}

\leavevmode\vadjust pre{\hypertarget{ref-norris1972}{}}%
Norris, K. S., \& Harvey, G. W. (1972, January 1). \emph{A theory for
the function of the spermaceti organ of the sperm whale (physeter
catodon l.)}. Retrieved from
\url{https://ntrs.nasa.gov/citations/19720017437}

\leavevmode\vadjust pre{\hypertarget{ref-nosal2007}{}}%
Nosal, E.-M., \& Frazer, L. N. (2007). Sperm whale three-dimensional
track, swim orientation, beam pattern, and click levels observed on
bottom-mounted hydrophonesa). \emph{The Journal of the Acoustical
Society of America}, \emph{122}(4), 1969--1978.
\url{https://doi.org/10.1121/1.2775423}

\leavevmode\vadjust pre{\hypertarget{ref-officeofnationalmarinesanctuaries}{}}%
Office of National Marine Sanctuaries. (n.d.). \emph{West coast region}.
Retrieved from \url{https://sanctuaries.noaa.gov/about/westcoast.html}

\leavevmode\vadjust pre{\hypertarget{ref-oldach2022}{}}%
Oldach, E., Killeen, H., Shukla, P., Brauer, E., Carter, N., Fields, J.,
\ldots{} Fangue, N. (2022). Managed and unmanaged whale mortality in the
california current ecosystem. \emph{Marine Policy}, \emph{140}, 105039.
\url{https://doi.org/10.1016/j.marpol.2022.105039}

\leavevmode\vadjust pre{\hypertarget{ref-parsons2016}{}}%
Parsons, E. C. M. (2016). Why IUCN should replace {``}data deficient{''}
conservation status with a precautionary {``}assume threatened{''}
status{\textemdash}a cetacean case study. \emph{Frontiers in Marine
Science}, \emph{3}. Retrieved from
\url{https://www.frontiersin.org/articles/10.3389/fmars.2016.00193}

\leavevmode\vadjust pre{\hypertarget{ref-peuxf1a2007}{}}%
Peña, M. A., \& Bograd, S. J. (2007). Time series of the northeast
pacific. \emph{Progress in Oceanography}, \emph{75}(2), 115--119.
\url{https://doi.org/10.1016/j.pocean.2007.08.008}

\leavevmode\vadjust pre{\hypertarget{ref-pijanowski2011}{}}%
Pijanowski, B. C., Farina, A., Gage, S. H., Dumyahn, S. L., \& Krause,
B. L. (2011). What is soundscape ecology? An introduction and overview
of an emerging new science. \emph{Landscape Ecology}, \emph{26}(9),
1213--1232. \url{https://doi.org/10.1007/s10980-011-9600-8}

\leavevmode\vadjust pre{\hypertarget{ref-pirotta2012}{}}%
Pirotta, E., Milor, R., Quick, N., Moretti, D., Marzio, N. D., Tyack,
P., \ldots{} Hastie, G. (2012). Vessel Noise Affects Beaked Whale
Behavior: Results of a Dedicated Acoustic Response Study. \emph{PLOS
ONE}, \emph{7}(8), e42535.
\url{https://doi.org/10.1371/journal.pone.0042535}

\leavevmode\vadjust pre{\hypertarget{ref-pitman}{}}%
Pitman, R. L., \& Brownell, R. L. (n.d.). Mesoplodon carlhubbsi.
\emph{The IUCN Red List of Threatened Species 2020: E.T13243A50364109}.
https://doi.org/\url{https://dx.doi.org/10.2305/IUCN.UK.2020-3.RLTS.T13243A50364109.en.}

\leavevmode\vadjust pre{\hypertarget{ref-pitman2020}{}}%
Pitman, R. L., \& Brownell, R. L. (2020). Mesoplodon ginkgodens.
\emph{The IUCN Red List of Threatened Species 2020:
E.T127827012A127827154}.
https://doi.org/\url{https://dx.doi.org/10.2305/IUCN.UK.2020-3.RLTS.T127827012A127827154.en.}

\leavevmode\vadjust pre{\hypertarget{ref-radford2010}{}}%
Radford, C. A., Stanley, J. A., Tindle, C. T., Montgomery, J. C., \&
Jeffs, A. G. (2010). Localised coastal habitats have distinct underwater
sound signatures. \emph{Marine Ecology Progress Series}, \emph{401},
21--29. \url{https://doi.org/10.3354/meps08451}

\leavevmode\vadjust pre{\hypertarget{ref-rand2022}{}}%
Rand, Z. R., Wood, J. D., \& Oswald, J. N. (2022). Effects of duty
cycles on passive acoustic monitoring of southern resident killer whale
(orcinus orca) occurrence and behavior. \emph{The Journal of the
Acoustical Society of America}, \emph{151}(3), 1651--1660.
\url{https://doi.org/10.1121/10.0009752}

\leavevmode\vadjust pre{\hypertarget{ref-reimer2016}{}}%
Reimer, J., Gravel, C., Brown, M. W., \& Taggart, C. T. (2016).
Mitigating vessel strikes: The problem of the peripatetic whales and the
peripatetic fleet. \emph{Marine Policy}, \emph{68}, 91--99.
\url{https://doi.org/10.1016/j.marpol.2016.02.017}

\leavevmode\vadjust pre{\hypertarget{ref-richardson2013}{}}%
Richardson, W. J., Jr, C. R. G., Malme, C. I., \& Thomson, D. H. (2013).
\emph{Marine Mammals and Noise}. Academic Press.

\leavevmode\vadjust pre{\hypertarget{ref-robbins2022}{}}%
Robbins, J. R., Bell, E., Potts, J., Babey, L., \& Marley, S. A. (2022).
Likely year-round presence of beaked whales in the Bay of Biscay.
\emph{Hydrobiologia}, \emph{849}(10), 2225--2239.
\url{https://doi.org/10.1007/s10750-022-04822-y}

\leavevmode\vadjust pre{\hypertarget{ref-ryther1969}{}}%
Ryther, J. H. (1969). Photosynthesis and fish production in the sea.
\emph{Science}, \emph{166}(3901), 72--76. Retrieved from
\url{https://www.jstor.org/stable/1727735}

\leavevmode\vadjust pre{\hypertarget{ref-sebastianutto2015}{}}%
Sebastianutto, L., Fortuna, C. M., Mackelworth, P., Holcer, D., \& Rako
Gospić, N. (2015, November 27). \emph{Popper, a. N. And hawkins, a. Eds.
(2015). The effects of noise on aquatic life II. Springer
science+business media, LLC, new york.} \emph{875}.
\url{https://doi.org/10.1007/978-1-4939-2981-8_101}

\leavevmode\vadjust pre{\hypertarget{ref-shannon2016}{}}%
Shannon, G., McKenna, M. F., Angeloni, L. M., Crooks, K. R., Fristrup,
K. M., Brown, E., \ldots{} Wittemyer, G. (2016). A synthesis of two
decades of research documenting the effects of noise on wildlife.
\emph{Biological Reviews}, \emph{91}(4), 982--1005.
\url{https://doi.org/10.1111/brv.12207}

\leavevmode\vadjust pre{\hypertarget{ref-simonis2020}{}}%
Simonis, A. E., Brownell, R. L., Thayre, B. J., Trickey, J. S., Oleson,
E. M., Huntington, R., \& Baumann-Pickering, S. (2020). Co-occurrence of
beaked whale strandings and naval sonar in the mariana islands, western
pacific. \emph{Proceedings of the Royal Society B: Biological Sciences},
\emph{287}(1921), 20200070. \url{https://doi.org/10.1098/rspb.2020.0070}

\leavevmode\vadjust pre{\hypertarget{ref-simonis2020a}{}}%
Simonis, A. E., Trickey, J. S., Barlow, J. P., Rankin, S., Urban, J.,
Rojas-Bracho, L., \& Moore, J. E. (2020). \emph{Passive acoustic survey
of deep-diving odontocetes in the california current ecosystem 2018.}

\leavevmode\vadjust pre{\hypertarget{ref-sivle2012}{}}%
Sivle, L., Kvadsheim, P., Fahlman, A., Lam, F.-P., Tyack, P., \& Miller,
P. (2012). Changes in dive behavior during naval sonar exposure in
killer whales, long-finned pilot whales, and sperm whales.
\emph{Frontiers in Physiology}, \emph{3}. Retrieved from
\url{https://www.frontiersin.org/articles/10.3389/fphys.2012.00400}

\leavevmode\vadjust pre{\hypertarget{ref-southall2017}{}}%
Southall, B. L., Scholik-Schlomer, A. R., Hatch, L., Bergmann, T.,
Jasny, M., Metcalf, K., \ldots{} Wright, A. J. (2017). \emph{Underwater
Noise from Large Commercial Ships{\textemdash}International
Collaboration for Noise Reduction}.
\url{https://doi.org/10.1002/9781118476406.emoe056}

\leavevmode\vadjust pre{\hypertarget{ref-stanistreet2022}{}}%
Stanistreet, J. E., Beslin, W. A. M., Kowarski, K., Martin, S. B.,
Westell, A., \& Moors-Murphy, H. B. (2022). Changes in the acoustic
activity of beaked whales and sperm whales recorded during a naval
training exercise off eastern Canada. \emph{Scientific Reports},
\emph{12}(1), 1973. \url{https://doi.org/10.1038/s41598-022-05930-4}

\leavevmode\vadjust pre{\hypertarget{ref-symonds2011}{}}%
Symonds, M. R. E., \& Moussalli, A. (2011). A brief guide to model
selection, multimodel inference and model averaging in behavioural
ecology using Akaike{'}s information criterion. \emph{Behavioral Ecology
and Sociobiology}, \emph{65}(1), 13--21.
\url{https://doi.org/10.1007/s00265-010-1037-6}

\leavevmode\vadjust pre{\hypertarget{ref-taylor}{}}%
Taylor, B. L., Baird, R. W., Barlow, J., Dawson, S. M., Ford, J., Mead,
J. G., \ldots{} Pitman, R. L. (n.d.). Physeter macrocephalus (amended
version of 2008 assessment). \emph{The IUCN Red List of Threatened
Species 2019: E.T41755A160983555}.
https://doi.org/\url{https://dx.doi.org/10.2305/IUCN.UK.2008.RLTS.T41755A160983555.en}

\leavevmode\vadjust pre{\hypertarget{ref-tyack2008}{}}%
Tyack, P. L. (2008). Implications for marine mammals of large-scale
changes in the marine acoustic environment. \emph{Journal of Mammalogy},
\emph{89}(3), 549--558. Retrieved from
\url{https://www.proquest.com/docview/221473276/abstract/CBFFD0F9B5AA4C54PQ/1}

\leavevmode\vadjust pre{\hypertarget{ref-tyack2006}{}}%
Tyack, P. L., Aguilar Soto, N., Johnson, M., Sturlese, A., \& Madsen, P.
T. (2006). Extreme diving of beaked whales \textbar{} journal of
experimental biology \textbar{} the company of biologists. \emph{Journal
of Experimental Biology}, \emph{209}, 4238--4253.
\url{https://doi.org/10.1242/jeb.02505}

\leavevmode\vadjust pre{\hypertarget{ref-unctad2022}{}}%
UNCTAD. (2022). \emph{Review of maritime transport 2022}.

\leavevmode\vadjust pre{\hypertarget{ref-unitedstatesarmycorpsofengineersusace2018}{}}%
United States Army Corps of Engineers {[}USACE{]}. (2018). \emph{The
u.s. Coastal and inland navigation system 2021 transportation facts \&
information}.

\leavevmode\vadjust pre{\hypertarget{ref-urick1983}{}}%
Urick, R. J. (1983). \emph{Principles of underwater sound} (3rd ed.).
New York : McGraw-Hill.

\leavevmode\vadjust pre{\hypertarget{ref-vagle2021}{}}%
Vagle, S., Burnham, R. E., O'Neill, C., \& Yurk, H. (2021). Variability
in Anthropogenic Underwater Noise Due to Bathymetry and Sound Speed
Characteristics. \emph{Journal of Marine Science and Engineering},
\emph{9}(10), 1047. \url{https://doi.org/10.3390/jmse9101047}

\leavevmode\vadjust pre{\hypertarget{ref-virgili2019}{}}%
Virgili, A., Authier, M., Boisseau, O., Cañadas, A., Claridge, D., Cole,
T., \ldots{} Ridoux, V. (2019). Combining multiple visual surveys to
model the habitat of deep-diving cetaceans at the basin scale.
\emph{Global Ecology and Biogeography}, \emph{28}(3), 300--314.
\url{https://doi.org/10.1111/geb.12850}

\leavevmode\vadjust pre{\hypertarget{ref-virgili2022}{}}%
Virgili, A., Teillard, V., Dorémus, G., Dunn, T. E., Laran, S., Lewis,
M., \ldots{} Ridoux, V. (2022). Deep ocean drivers better explain
habitat preferences of sperm whales Physeter macrocephalus than beaked
whales in the Bay of Biscay. \emph{Scientific Reports}, \emph{12}(1),
9620. \url{https://doi.org/10.1038/s41598-022-13546-x}

\leavevmode\vadjust pre{\hypertarget{ref-watkins1993}{}}%
Watkins, W. A., Daher, M. A., Fristrup, K. M., Howald, T. J., \& Di
Sciara, G. N. (1993). Sperm Whales Tagged with Transponders and Tracked
Underwater by Sonar. \emph{Marine Mammal Science}, \emph{9}(1), 55--67.
\url{https://doi.org/10.1111/j.1748-7692.1993.tb00426.x}

\leavevmode\vadjust pre{\hypertarget{ref-weatherall2015}{}}%
Weatherall, P., Marks, K. M., Jakobsson, M., Schmitt, T., Tani, S.,
Arndt, J. E., \ldots{} Wigley, R. (2015). A new digital bathymetric
model of the world's oceans. \emph{Earth and Space Science},
\emph{2}(8), 331--345. \url{https://doi.org/10.1002/2015EA000107}

\leavevmode\vadjust pre{\hypertarget{ref-weilgart2018}{}}%
Weilgart, L. (2018). \emph{The impact of ocean noise pollution on fish
and invertebrates} (p. 34).

\leavevmode\vadjust pre{\hypertarget{ref-weilgart2007}{}}%
Weilgart, L. S. (2007). The impacts of anthropogenic ocean noise on
cetaceans and implications for management. \emph{Can. J. Zool.},
\emph{85}, 10911116. Retrieved from
\url{https://cdnsciencepub.com/doi/abs/10.1139/z07-101}

\leavevmode\vadjust pre{\hypertarget{ref-weiss2021}{}}%
Weiss, S. G., Cholewiak, D., Frasier, K. E., Trickey, J. S.,
Baumann-Pickering, S., Hildebrand, J. A., \& Van Parijs, S. M. (2021).
Monitoring the acoustic ecology of the shelf break of Georges Bank,
Northwestern Atlantic Ocean: New approaches to visualizing complex
acoustic data. \emph{Marine Policy}, \emph{130}, 104570.
\url{https://doi.org/10.1016/j.marpol.2021.104570}

\leavevmode\vadjust pre{\hypertarget{ref-west2017}{}}%
West, K. L., Walker, W. A., Baird, R. W., Mead, J. G., \& Collins, P. W.
(2017). Diet of cuvier{'}s beaked whales ziphius cavirostris from the
north pacific and a comparison with their diet world-wide. \emph{Marine
Ecology Progress Series}, \emph{574}, 227--242. Retrieved from
\url{https://www.jstor.org/stable/26403641}

\leavevmode\vadjust pre{\hypertarget{ref-winsor2017}{}}%
Winsor, M. H., Irvine, L. M., \& Mate, B. R. (2017). Analysis of the
spatial distribution of satellite-tagged sperm whales (physeter
macrocephalus) in close proximity to seismic surveys in the gulf of
mexico. \emph{Aquatic Mammals}, \emph{43}(4), 439--446.
\url{https://doi.org/10.1578/AM.43.4.2017.439}

\leavevmode\vadjust pre{\hypertarget{ref-yamada2019}{}}%
Yamada, T. K., Kitamura, S., Abe, S., Tajima, Y., Matsuda, A., Mead, J.
G., \& Matsuishi, T. F. (2019). Description of a new species of beaked
whale (Berardius) found in the North Pacific. \emph{Scientific Reports},
\emph{9}(1), 12723. \url{https://doi.org/10.1038/s41598-019-46703-w}

\leavevmode\vadjust pre{\hypertarget{ref-ziegenhorn2023}{}}%
Ziegenhorn, M. A., Hildebrand, J. A., Oleson, E. M., Baird, R. W.,
Wiggins, S. M., \& Baumann-Pickering, S. (2023). Odontocete spatial
patterns and temporal drivers of detection at sites in the Hawaiian
islands. \emph{Ecology and Evolution}, \emph{13}(1), e9688.
\url{https://doi.org/10.1002/ece3.9688}

\end{CSLReferences}

\end{document}
